\chapter*{چکیده}
\baselineskip=.90cm

گنو/لینوکس در رایانه‌های کارساز وب سیستم‌عامل کاملا موفقی به شمار می‌آید و اکثر رایانه‌های کارساز وب مشغول استفاده از گنو/لینوکس هستند. اگر  سیستم‌عامل گنو/لینوکس (برخی افراد به نام سیستم‌عامل لینوکس می‌شناسند) به همراه ابزار کارساز وب آپاچی،ابزار  مدیریت بانک اطلاعاتی مای‌اس‌کی‌یو‌ال (برخی موارد ماریا‌دی‌بی و زبان برنامه‌نویسی پی‌اچ‌پی را کنار هم قرار دهیم و در یک کارساز شبکه از آن استفاده کنیم اصطلاحا یک {\lr{LAMP}} ایجاد کرده‌ایم. این نام مخفف کلمات اول لینوکس، مای‌اس‌کی‌یو‌ال، آپاچی و زبان برنامه‌نویسی تحت وب پی‌اچ‌پی (حتی برای پرل و پایتون هم تقریبا می شود به کاربرد زیرا اول هرکدام  با حرف پی لاتین شروع شده) است.

امروزه اینترنت موجب گسترش دانش و نشر دانش شده است به صورتی که اکثر افراد می‌توانند داشته‌ها و دانسته‌های خود را در این بستر ارتباطی و رسانهٔ که خود از قابلیت پوشش دادن به نیازهای چندرسانه‌ای کاربران را دارد، با دیگران در میان بگزارند. این مجموعهٔ آموزشی که در حال حاضر در قسمت آخر آن هستیم نیز برای این موضوع نوشته شده است، تا کاربرانی که قصد دارند به توسعهٔ نرم‌افزار تحت وب یا طراحی صفحات وب پردازند، با اندک هزینه‌ای محیطی ارزان و ساده را برای اجرای نرم‌افزار و کدهای خود ایجاد کنند.  با استفاده از این محیط سند باکس به راحتی می‌توان تمامی پروژه‌های مختلف را اجرا کرد.
با این وجود اگر از یک رزبری‌پای نیز استفاده می‌کنید، می‌توانید با استفاده از یک سوئیچ یا یک مودم که در منزل دارید، شبکه‌ای بی‌سیم در منزل خود ایجاد کنید، و رزبری‌پای را به آن متصل کنید، سپس اوبونتو سرور را بر اساس آموزشی که در این مطلب داده شد نصب کرده و با استفاده از انتقال درگاه یا پورت فورواردینگ، درگاه‌هایی را که در ویرچوال‌باکس انتقال دادیم، به صورتی انتقال دهید که در تمامی شبکهٔ محلی خود بتوانید به آنان متصل شوید.  بعد از این شما از یک محیط توسعه برخوردار می‌شوید که هر وقت و هرگاه خواستید می‌توانید به آن متصل شوید. البته در این حالت برای ذخیره اطلاعات پر حجم به یک دیسک سخت با ظرفیت مناسب نیز نیاز خواهید داشت.

اگر در مورد آموزش فوق، نحوه قسمت‌بندی مطالب و قسمت‌های عنوان شده تدر کتاب ایرادی را مشاهده می‌کنید، در قسمت نظرات و یا تماس با من از طریق رایانامه و شبکه‌های اجتماعی موارد قابل ذکر را متذکر شوید، تا در قسمت‌های بعدی یا دیگر مقالات رفع شوند. همچنین ایرادات دیگر در این مقالات را با رایانامه برای من بفرستید تا این ایرادات رفع شوند. این ایرادات می‌تواند در نحوه نوشتن کدها و انجام تنظیمات وجود داشته باشند که در صورت وجود چنین ایرادتی در کدها و تنظیمات، آنان را رفع کنم. اگر خواستید کدها و تنظیمات فوق را در اختیار داشته باشید نیز از طریق رایانامه، درخواست خود را ارسال کنید، تا اکثر فایل‌ها را برایتان ارسال کنم. گفتنی است سیستم‌عامل مورد استفاده من در این مقاله گنو/لینوکس توزیع آرچ‌لینوکس است.

