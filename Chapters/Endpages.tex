%%%%%%%%%%%
% آوردن مراجع در انتهای گزارش با فرمت IEEE 
% فرمت مراجع می تواند برای ACM و ... نیز باشد. برای این کار کافی است تنها 
%پارامتر style را باید تغییر دهید. 
\baselineskip=.65cm

% مراجع همگی در یک فایل bibtex با پسوند .bib  وجود دارد، که می بایست در پوشه اصلی گزارش قرار داده شود.
% نام این فایل می بایست library باشد. در غیر این صورت باید نام فایل bib را در خط زیر تغییر دهید. 
%\renewcommand{\bibname}{مراجع}
\clearpage
\phantomsection
%\addcontentsline{toc}{chapter}{کتاب نامه}
\bibliographystyle{plain-fa}{persia}
\bibliography{library}




%%%%%%%%%%
\baselineskip=.75cm
%\glossarystyle{IEEE}
\clearpage
\phantomsection
\def\glossaryname{واژه نامه فارسی به انگلیسی}
\addcontentsline{toc}{chapter}{\glossaryname}
\printglossary[title={ \glossaryname}]

% قرار دادن واژه نامه انگلیسی به فارسی

\begin{LTR}
\def\glossaryname{\rl{واژه نامه انگلیسی به فارسی}}
%\addcontentsline{toc}{chapter}{\glossaryname}
\begin{LTR}
\printglossary[type={mylist1},title={\begin{center} \glossaryname \end{center}}]
\end{LTR}
\end{LTR}



%%%%%%%%%%
% قرار دادن نمایه کلمات به عنوان آخرین قسمت گزارش
\baselineskip=.75cm
\clearpage
\phantomsection
\addcontentsline{toc}{chapter}{نمایه}
\printindex
\clearpage

